% !TEX TS-program = latexmk

\documentclass[12pt,twoside,spanish]{book}

% Verificacion de sintaxis sin producir salida:
% descomentar la segunda linea para verificar
% sintaxis sin producir salida.
\usepackage{syntonly}
% \syntaxonly

% Palabras clave en distintos idiomas
\usepackage[activeacute]{babel}

% Patrones para division en silabas
\hyphenation{par-ti-cu-lar}

% Inclusion de imagenes
\usepackage{graphicx}

% Codificacion de simbolos
\usepackage[T1]{fontenc}
\usepackage[applemac]{inputenc}

% Fuentes y simbolos adicionales
\usepackage{amsfonts}
\usepackage{amsmath}
\usepackage{amssymb}
\usepackage{mathrsfs}

% Colores
\usepackage[table]{xcolor}

\definecolor{verdeOscuro}{rgb}{0,0.6,0}
\definecolor{verdeAzulado}{RGB}{3,168,108}
\definecolor{gris}{rgb}{0.5,0.5,0.5}
\definecolor{malva}{rgb}{0.58,0,0.82}
\definecolor{grisClaro}{rgb}{0.92,0.92,0.92}

% Vinculos
\usepackage{hyperref}
\hypersetup{
  colorlinks,
  urlcolor={malva}
}

% Codigo
\usepackage{listings}
\lstset{
  language=[LaTeX]TeX,
  aboveskip=3mm,
  belowskip=3mm,
  showstringspaces=false,
  columns=flexible,
  basicstyle={\small\ttfamily},
  numbers=none,
  extendedchars=true,
  numberstyle=\tiny\color{gris},
  keywordstyle=\color{blue},
  commentstyle=\color{verdeOscuro},
  stringstyle=\color{malva},
  breaklines=true,
  breakatwhitespace=true,
  tabsize=3,
  backgroundcolor=\color{grisClaro},
  moretexcs={color},
}


% Algoritmos
\usepackage[ruled,lined,linesnumbered,commentsnumbered]{algorithm2e}

%\usepackage[all]{xy}

% Generacion de dibujos con TeX
\usepackage{tikz}
\usetikzlibrary{arrows,shapes,matrix,decorations,calc,positioning}
\tikzstyle{arc}   =[->,shorten <=3pt, shorten >=3pt,
                   >=stealth, line width=1.1pt]
\tikzstyle{edge}  =[shorten <=2pt, shorten >=2pt,
                    >=stealth, line width=1.1pt]
\tikzstyle{myloop}=[style={},shorten <=1pt, shorten >=1pt,
                    >=stealth, line width=1.1pt, loop]
\tikzstyle{vertex}=[circle, draw, minimum size=6pt,
                    line width=0.75pt, inner sep=0pt,
                    outer sep=0pt]


% Ajuste de parametros en la pagina
\usepackage{geometry}

% Ubicacion precisa de floats
\usepackage{float}

% Manejo de encabezados
\usepackage{fancyhdr}

% Estilo de encabezados
\pagestyle{fancy}
\renewcommand{\chaptermark}[1]{\markboth{#1}{}}
\renewcommand{\sectionmark}[1]{\markright{\thesection\ #1}}
\fancyhf{}
\fancyhead[LE,RO]{\textsc{\bfseries\nouppercase\thepage}}
\fancyhead[LO]{\textsc{\bfseries\nouppercase\rightmark}}
\fancyhead[RE]{\textsc{\bfseries\nouppercase\leftmark}}
\renewcommand{\headrulewidth}{0.5pt}
\renewcommand{\footrulewidth}{0pt}
\addtolength{\headheight}{3pt}
\fancypagestyle{plain}{
  \fancyhead{}
  \renewcommand{\headrulewidth}{0pt}
}

% Delimitadores para terminar las demostraciones
\newcommand{\blackqed}{\hfill$\blacksquare$}
\newcommand{\whiteqed}{\hfill$\square$}
\newcounter{proofcount}

% Generacion de tabla de contenidos
\usepackage{makeidx}
\makeindex

% Ambientes de teorema y demostracion
\usepackage{amsthm}

% Referencias con nombres automaticos
\usepackage{cleveref}

% Definiciones de nuevos ambientes
\newtheorem{teorema}{Teorema}[section]
\newtheorem{lema}[teorema]{Lema}
\newtheorem{proposicion}[teorema]{Proposici\'on}
\newtheorem{corolario}[teorema]{Corolario}

\theoremstyle{definition}
\newtheorem{definicion}[teorema]{Definici\'on}

% Nombres para cleveref
\crefname{teorema}{el Teorema}{los Teoremas}
\crefname{lema}{el Lema}{los Lemas}
\crefname{proposicion}{la Proposici\'on}{las Proposiciones}
\crefname{corolario}{el Corolario}{los Corolarios}
\crefname{algorithm}{el Algoritmo}{los Algoritmos}
\crefname{section}{la Secci\'on}{las Secciones}
\crefname{figure}{la Figura}{las Figuras}

% Redefinicion para que las demostraciones terminen con cuadrito negro
\renewenvironment{proof}[1][\proofname.]{\par
\ifnum \theproofcount>0 \pushQED{\whiteqed} \else \pushQED{\blackqed} \fi%
\refstepcounter{proofcount}
%
\normalfont %\topsep6\p@\@plus6\p@\relax
\trivlist
\item[\hskip\labelsep
\itshape
\textbf{\textit{#1}}]\ignorespaces
}{%
\addtocounter{proofcount}{-1}
\popQED\endtrivlist
}

% Macros
% Abreviatura para fuentes true type
\newcommand{\ttt}[1]{%
\texttt{#1}%
}

\newcommand{\indice}[1]{%
\textbf{#1}\index{#1}%
}

\newcommand{\indiceSub}[2]{%
\textbf{#2}\index{#1!#2}%
}

\begin{document}

\frontmatter

%%
%% Portada creada por la Facultad de Ciencias
%%

%%
%% Esqueleto para la portada de las tesis para
%% la Facultad de Ciencias de la UNAM
%%


%%%%%%%%%%%%%%%%%%%%%%%%%%%%%%%%%%%%%%%%%%%%%%%%%%%%%%%
%% Comandos para la portada

\newcommand{\titulo}[1]{\def\eltitulo{#1}}
%* la carrera corresponde al tí­tulo otorgado -Matemático-
%* NO AL NOMBRE DE LA CARRERA -Matemáticas- --RRP
\newcommand{\carrera}[1]{\def\lacarrera{#1}}
\newcommand{\nombre}[1]{\def\elnombre{#1}}      %* Del alumno
\newcommand{\director}[1]{\def\eldirector{#1}}  %* De tesis
\newcommand{\fecha}[1]{\def\lafecha{#1}}

%% Llene los siguientes datos (use MAYÚSCULAS)
%% Estos datos aparecerán en la portada y los encabezados
\titulo{T\'ITULO DE LA TESIS}
\nombre{\uppercase{NOMBRE COMPLETO}}
\carrera{T\'ITULO OTORGADO}
\director{\uppercase{C\'ESAR HERN\'ANDEZ CRUZ}}
\fecha{2021}


\thispagestyle{empty}

%% Barra izquierda - Escudos
\hskip-1.5cm
\begin{minipage}[c][10cm][s]{3cm}
  \begin{center}
    \includegraphics[height=2.6cm]{escudo-unam}\\[10pt]
    \hskip2pt\vrule width2pt height13cm\hskip1mm
    \vrule width1pt height13cm\\[10pt]
    \includegraphics[height=2.6cm]{escudo-ciencias}
  \end{center}
\end{minipage}\quad
%% Barra derecha - Tí­tulos
\begin{minipage}[c][9.5cm][s]{10cm}
  \begin{center}
    % Barra superior
    {\large \scshape Universidad Nacional Aut\'onoma de M\'exico}
    \vspace{.3cm}
    \hrule height2pt
    \vspace{.1cm}
    \hrule height1pt
    \vspace{.3cm}
    {\scshape Facultad de Ciencias}

    % Titulo del trabajo
    \vspace{3cm}

    {\Large \eltitulo}

    \vspace{3cm}

    % Tipo de trabajo
    \makebox[8cm][s]{\Huge T E S I S}\\[8pt]
    QUE PARA OBTENER EL T\'ITULO DE:\\[3pt]
    \mbox{}\lacarrera\\[13pt]
    PRESENTA:\\[3pt]
    \elnombre

    \vspace{2cm}

    {\small DIRECTOR DE TESIS:\\ \eldirector}

    \vspace{2cm}

    \lafecha

  \end{center}
\end{minipage}

{\small
\begin{quote}
\begin{tabular}{lll}
1.Datos del alumno          & {}                                          \\
Apellido paterno            & Paterno                                     \\
Apellido materno            & Materno                                     \\
Nombre(s)                   & Nombres                                     \\
Tel\'efono                  & XX XX XX XX XX                              \\
Universidad                 & Universidad Nacional Aut\'onoma de M\'exico \\
Facultad o escuela          & Facultad de Ciencias                        \\
Carrera                     & Carrera                                     \\
N\'umero de cuenta          & XXXXXXXXX                                   \\
{}                          & {}                                          \\
2. Datos del tutor          & {}                                          \\
Grado                       & Dr.                                         \\
Nombre(s)                   & C\'esar                                     \\
Apellido paterno            & Hern\'andez                                 \\
Apellido materno            & Cruz                                        \\
{}                          & {}                                          \\
3. Datos del sinodal 1      & {}                                          \\
Grado                       & Dra.                                        \\
Nombre(s)                   & Nombres                                     \\
Apellido paterno            & Paterno                                     \\
Apellido materno            & Materno                                     \\
{}                          & {}                                          \\
4. Datos del sinodal 2      & {}                                          \\
Grado                       & M. en C.                                    \\
Nombre(s)                   & Nombres                                     \\
Apellido paterno            & Paterno                                     \\
Apellido materno            & Materno                                     \\
{}                          & {}                                          \\
5. Datos del sinodal 3      & {}                                          \\
Grado                       & Mat.                                        \\
Nombre(s)                   & Nombres                                     \\
Apellido paterno            & Paterno                                     \\
Apellido materno            & Materno                                     \\
{}                          & {}                                          \\
6. Datos del sinodal 4      & {}                                          \\
Grado                       & Lic. en Ciencias de la Computaci\'on        \\
Nombre(s)                   & Nombres                                     \\
Apellido paterno            & Paterno                                     \\
Apellido materno            & Materno                                     \\
{}                          & {}                                          \\
7.Datos del trabajo escrito & {}                                          \\
T\'itulo                    & T\'itulo                                    \\
N\'umero de p\'aginas       & XX p.                                       \\
A\~no                       & 2021                                        \\
\end{tabular}
\end{quote}
}

\chapter*{Agradecimientos}
\addcontentsline{toc}{chapter}{Agradecimientos}

Agradezco a mi director de tesis por haber hecho esta plantilla.

\tableofcontents


\mainmatter

\chapter{Introducci\'on}
\label{sec:intro}

Este documento tiene algunos ejemplos m\'inimos de caracter\'isticas que se
suelen utilizar en tesis de las licenciaturas en matem\'aticas y ciencias de la
computaci\'on, en particular en el \'area de teor\'ia de gr\'aficas (el \'area
de trabajo del autor).

Se exhorta al usuario a leer la
\href{https://tobi.oetiker.ch/lshort/lshort.pdf}{Not So Short Inroduction to
\LaTeX}.   Aunque realmente no es un documento muy largo, para quienes nunca han
usado \LaTeX{} es posible que las partes t\'ecnicas no tengan sentido.   En este
caso, es recomendable leer los dos primeros cap\'itulos, y regresar al resto del
documento para hacer consultas, o cuando se tenga algo de experiencia y se
desee mejorar como usuario.   En particular, antes de intentar cambiar el tipo
de letra, o el tama\~no de los m\'argenes, considere la siguiente observaci\'on
que aparece en el documento antes mencionado:
\begin{quote}
  Typographical design  is  a  craft.   Unskilled  authors  often  commit
  seriousformatting errors  by  assuming  that  book  design  is  mostly  a
  question of aesthetics---``If a document looks good artistically, it is well
  designed.'' But as a document has to be read and not hung up in a picture
  gallery, the readability and understandability is of much greater importance
  than the beautiful look of it.
\end{quote}

Idealmente, el lector obtuvo esta plantilla mediante
\href{https://github.com/Japodrilo/template-tesis}{este repositorio}. De no ser
as\'i, se le invita a visitarlo, y a usar las bondades del control de versiones
que el uso de \href{https://git-scm.com/}{Git} otorga (en particular cuando se
utiliza en conjunto con alguna plataforma para albergar sus repositorios
remotamente\footnote{Los alumnos de la Facultad de Ciencias de la UNAM tienen
acceso al \href{https://education.github.com/pack}{GitHub Student Developer
Pack} con su cuenta \ttt{@ciencias.unam.mx}.}).


\section{C\'omo usar esta plantilla}
\label{sec:howto}

Esta plantilla se dise\~n\'o como una ayuda para aquellos usuarios que ya
est\'an familiarizados con \LaTeX, pero nunca han desarrollado un proyecto
``grande'' (m\'as all\'a de tareas o reportes finales de proyectos).   Siguiendo
las instrucciones encontradas en el archivo \ttt{README.md}, lo m\'as probable
es que hayan creado un nuevo repositorio a partir del ``template repository''
que contiene este proyecto.   En primer lugar, verifique que el proyecto compile
adecuadamente; el proyecto deber\'ia de compilar sin errores ni advertencias. Es
posible que la primera vez que se compila, su manejador de paquetes actualice
varios de \'estos, lo que puede llevar un tiempo.   En caso de tener errores, es
posible que \'estos se deban a la falta de algunos paquetes, y a que su
manejador de paquetes no los instala autom\'aticamente;  instalar los paquetes
faltantes manualmente deber\'ia de resolver todos los problemas.

La estructura de este proyecto es sencilla.   Hay un archivo central,
\ttt{tesis.tex}, que contiene el pre\'ambulo del documento, y donde se incluyen
todos los paquetes y definiciones necesarias.   El c\'odigo esta comentado,
explicando de forma m\'inima para qu\'e sirve cada comando; se recomienda que al
modificarlo se mantenga un estilo semejante para no causarle problemas
innecesarios a su yo del futuro.   Todos los contenidos se encuentran en otros
archivos dentro del mismo directorio, que son llamados desde \ttt{tesis.tex}
mediante el comando \ttt{\textbackslash{include}}.   De esta forma se
incluyen la car\'atula, la hoja de datos, los cap\'itulos que forman parte de la
tesis, la bibliograf\'ia, etc.   Por otro lado, \LaTeX genera (casi)
autom\'aticamente el \'indice y el \'indice alfab\'etico, pero hay que agregar
comandos para su inclusi\'on.  La mayor\'ia de los usuarios s\'olo necesitan
preocuparse por modificar algunos de los archivos existentes, e incluir otros.
Sin embargo, es \'util que est\'en familiarizados con los conceptos de
\ttt{frontmatter}, \ttt{mainmatter}, \ttt{appendix} y \ttt{backmatter} (puede
referirse a \cite{oetiker2007} para revisarlos).

A continuaci\'on, se recomienda revisar el documento generado (este documento) e
identificar cu\'ales son las caracter\'isticas que se desean utilizar (dibujos,
algoritmos, tablas, etc.).   Tras determinar cu\'ales son los paquetes
relevantes para las caracter\'isticas deseadas, comentar (o borrar) todos
aquellos que no ser\'an utilizados en el archivo \ttt{tesis.tex}.   Si se est\'a
usando \ttt{git}, se recomienda leer \cref{sec:git}.   De otro modo, puede
empezar a reemplazar los contenidos de la plantilla con su propio trabajo.

\section[Uso recomendado con git]{Flujo de trabajo recomendado con \ttt{git}}
\label{sec:git}

Si el lector no est\'a usando \ttt{git}\index{git}, puede ignorar esta
secci\'on.   De otro modo, se propone un flujo de trabajo con el que el tesista
puede autogestionar el desarrollo de su tesis, o \'este puede ser supervisado
por su director de tesis mediante el uso de \ttt{GitHub}\index{git!GitHub}.

Este repositorio cuenta con tres ramas al momento de ser clonado: \ttt{master},
\ttt{original} y \ttt{prueba}.   Idealmente, \ttt{master} debe contener su
trabajo final, una vez que ha sido revisado por su director de tesis, por lo que
nunca deber\'ia de trabajar directamente sobre esta rama.   Por este motivo,
antes de realizar cambios y experimentos en los archivos del proyecto, se
recomienda cambiar a la rama \ttt{prueba}.   Tras realizar algunos experimentos,
eliminar los contenidos que no necesita, y agregar sus datos a la car\'atula y
hoja de datos, posiblemente se sienta listo para empezar a incluir su trabajo en
el proyecto.   En este momento se recomienda agregar los cambios realizados al
repositorio, realizar un \ttt{commit} con los mismos, y realizar un \ttt{merge}
a \ttt{master}.   A partir de ahora, \ttt{master} estar\'a lista para empezar a
trabajar.

En este momento, es posible crear
\href{https://guides.github.com/features/issues/}{\ttt{Issues}} en su
repositorio para tener metas de trabajo.   Como muy posiblemente s\'olo una
persona est\'e trabajando en el proyecto (el tesista), es posible que s\'olo
se trabaje en un \ttt{issue} a la vez, sin embargo, es una buena pr\'actica
tener una rama para cada \ttt{issue} (lo que resultar\'a a\'un m\'as \'util si
se trabaja en m\'as de una caracter\'istica nueva a la vez).   Idealmente, toda
rama nueva saldr\'a de \ttt{master}, y estar\'a dedicada a resolver un \'unico
\ttt{issue}.   Un ciclo de trabajo\index{ciclo de trabajo} usual puede verse
de la siguiente forma.

\begin{enumerate}
  \item Determinar una caracter\'istica nueva que se desea agregar al
    trabajo (e.g., la demostraci\'on de un teorema central de la tesis).

  \item Crear un \ttt{issue} describiendo qu\'e es lo que espera agregar
    al trabajo (e.g., qu\'e conceptos se necesitan agregar, proveer una
    referencia del teorema, indicar si es necesario incluir resultados
    preliminares o ejemplos).

  \item Asignar el \ttt{issue} al tesista, y opcionalmente agregar una fecha
    l\'imite.   (En caso de que el director de tesis est\'e supervisando el
    trabajo mediante \ttt{git}, asignarlo como revisor del \ttt{issue}.)

  \item Crear una nueva rama (a partir de \ttt{master}) para resolver el
    \ttt{issue}.

  \item Una vez resuelto el \ttt{issue}, hacer un \ttt{commit} con los cambios,
    un \ttt{push} al repositorio, y abrir un \ttt{pull request} que ser\'a
    cerrado una vez que el director de tesis haya revisado el nuevo trabajo.
    (En caso de que el director de tesis est\'e supervisando el trabajo mediante
    \ttt{git}, deber\'a de ser agregado como revisor del \ttt{pull request},
    y \'este ser\'a mezclado hasta tener su aprobaci\'on.)

  \item Tras aceptar el \ttt{pull request}, cerrar el \ttt{issue} y borrar la
    rama correspondientes (esto puede hacerse autom\'aticamente al aceptar el
    \ttt{pull request}).
\end{enumerate}

Un ciclo de trabajo tomar\'a tipicamente una semana, por lo que las metas a ser
cubiertas por cada \ttt{issue} deber\'an planearse con cuidado.

Se recomienda no modificar la rama \ttt{original}.   Si en cualquier momento se
necesitara tener acceso a este documento (quiz\'a el usuario requiere revisar un
ejemplo, o recuperar alg\'un paquete que borr\'o previamente), basta con
cambiarse a la rama \ttt{original}, donde siempre habr\'a una copia local del
mismo.

\chapter{Preliminares}


\section{Gr\'aficas}

Una \textbf{\textit{gr\'afica}}\index{gr\'afica}, $G$, es una terna ordenada $(V(G),E(G), \psi_G)$, compuesta por un
conjunto de v\'ertices, $V(G)$, un conjunto de aristas, $E(G)$, los cuales son ajenos entre si, y una funci\'on de
adyacencia, $\psi_G$, que le asocia a cada arista de $G$ un par no ordenado de v\'ertices (no necesariamente
distintos). Si $e$ es una arista y $u$ y $v$ son v\'ertices tales que $\psi_G(e) = \{u,v\}$, entonces se dice que $e$
une a $u$ y a $v$ y los v\'ertices $u$ y $v$ se llaman extremos de $e$.
Para denotar a la pareja no ordenada $\{u,v\}$ escribiremos $uv$.
Dos v\'ertices que son incidentes con una arista en com\'un se dice que son adyacentes o vecinos, asi como dos
aristas que son incidentes con un v\'ertice en com\'un. Al conjunto de v\'ertices adyacentes o vecinos a un v\'ertice
$v$ le llamamos vecindad de $v$ y la denotamos $N(v)$.
Al n\'umero de v\'ertices en G se le denota $v(G)$ y se le llama orden de $G$.
Al n\'umero de aristas en G se le denota $e(G)$ y se le llama tama\~no de $G$.

\begin{figure}[ht]
    \begin{center}
        \begin{tikzpicture}
        [every circle node/.style ={circle,draw,minimum size= 5pt,inner sep=0pt, outer sep=0pt},
            every rectangle node/.style ={}]
            ;

            \begin{scope}[scale=0.5]
                \node [circle] (1) at (90:3)[label=90:$v_2$]{};
                \node[circle] (2) at (210:3)[label=135:$v_1$]{};
                \node[circle] (3) at (330:3)[label=45:$v_3$]{};
                \node [circle] (4) at (4.5,2)[label=270:$v_4$]{};
                \node (g) at (-4.5,3.5)[]{$G$};


                %\draw [-, shorten <=3pt, shorten >=3pt, >=stealth, line width=.7pt] (2) to [bend right=30]
                %node[below] {$e_6$} (3);
                %\draw [-, shorten <=3pt, shorten >=3pt, >=stealth, line width=.7pt] (2) to node[left] {$e_1$} (1);
                %\draw [-, shorten <=3pt, shorten >=3pt, >=stealth, line width=.7pt] (2) to node[above] {$e_3$} (3);
                %\draw [-, shorten <=3pt, shorten >=3pt, >=stealth, line width=.7pt] (1) to node[right] {$e_2$} (3);
                %\draw [-, shorten <=3pt, shorten >=3pt, >=stealth, line width=.7pt] (1) to node[above] {$e_4$} (4);
                %\path [-, shorten <=3pt, shorten >=3pt, >=stealth, line width=.7pt, out=0, in=45, looseness=20, loop,
                %distance=3cm] (4) edge[loop left, bend right=50] node[right] {$e_5$} (4);
            \end{scope}


            \begin{scope}[xshift=7.5cm, scale=0.5]
                \node [circle] (1) at (90:3)[label=90:$v_2$]{};
                \node[circle] (2) at (210:3)[label=135:$v_1$]{};
                \node[circle] (3) at (330:3)[label=45:$v_3$]{};
                \node [circle] (4) at (4.5,2)[label=270:$v_4$]{};
                \node (g) at (-4.5,3.5)[]{$H$};

                %\draw [-, shorten <=3pt, shorten >=3pt, >=stealth, line width=.7pt] (2) to [bend right=30]
                %node[below] {$e_6$} (3);
                %\draw [-, shorten <=3pt, shorten >=3pt, >=stealth, line width=.7pt] (2) to node[left] {$e_1$} (1);
                %\draw [-, shorten <=3pt, shorten >=3pt, >=stealth, line width=.7pt] (1) to node[right] {$e_2$} (3);
                %\draw [-, shorten <=3pt, shorten >=3pt, >=stealth, line width=.7pt] (1) to node[above] {$e_4$} (4);
            \end{scope}


        \end{tikzpicture}
    \end{center}\caption{Ejemplo de una gr\'afica $G$=$(V(G),E(G))$ donde $V(G)$=$\{v_1,v_2,v_3,v_4\}$, $E(G)
    $=$\{e_1,e_2,e_3,e_4,e_5,e_6\}$, $\psi_G(e_1)$ = $v_1v_2$, $\psi_G(e_2)$ = $v_2v_3$, $\psi_G(e_3)$ = $v_1v_3$,
        $\psi_G(e_4)$ = $v_2v_4$, $\psi_G(e_5)$ = $v_4$ y $\psi_G(e_6)$ = $v_1v_3$. Adem\'as $H$ subgr\'afica de $G$,
        donde $H$=$(V(H), E(H))$, $V(H)$=$\{v_1, v_2, v_3, v_4\}$, $E(H)$=$\{e_1, e_2, e_4, e_6\}$ y
        $\psi_H$=$\psi_G\big|_{E(H)}$.}
\end{figure}


\index{subgr\'afica}
Una gr\'afica $H$ se llama \textbf{\textit{subgr\'afica}} de la gr\'afica $G$ si $V(H) \subseteq V(G)$, $E(H)
\subseteq E(G)$ y $\psi_H$ es la restricci\'on de $\psi_G$ a las aristas de $H$, es decir $\psi_H$=$\psi_G$$\big|_{E
(H)}$; entonces decimos que $G$ contiene a $H$. Si $X \subseteq V(G)$, la subgr\' afica de $G$ inducida por $X$ es la
gr\' afica $G[X]$ tal que $V(G[X]) = X$ y donde $uv \in E(G[X])$ si y s\'olo si $uv \in E(G)$.

Un\textbf{\textit{ lazo}} es una arista que tiene sus dos extremos iguales.
\index{gr\'afica!vac\'ia}
Una \textbf{\textit{gr\'afica vac\'ia}} es aquella que no tiene aristas.
Cuando la gr\'afica tiene un solo v\'ertice decimos que es una gr\'afica \textbf{\textit{trivial}} y en cualquier
otro caso decimos que es \textbf{\textit{no trivial}}.
\index{gr\'afica!finita}
Una \textbf{\textit{gr\'afica finita}} es en la que el conjunto de v\'ertices y aristas es finito.
\index{gr\'afica!simple}
Decimos que una gr\'afica es \textbf{\textit{simple}} si aristas distintas tienen extremos distintos y adem\'as no
tiene lazos.


En este trabajo nosotros s\'olo nos referiremos a gr\'aficas simples y a gr\'aficas finitas, entonces a partir de
este punto siempre que mencionemos una gr\'afica nos estaremos refiriendo a una gr\'afica simple y finita.

Por tanto observemos que para que $H$ sea subgr\'afica de $G$ basta ver que $V(H) \subseteq V(G)$ y $E(H) \subseteq E
(G)$ (observemos que ya que son gr\'aficas simples podemos omitir a la funci\'on de adyacencia y pensar a $G$ como
una pareja $G=(V,E)$ donde $V$ es un conjunto distinto al vacio y $E \subseteq  [V]^2 = \left\{ S\subseteq V\big| |s|
=2\right\}$, es decir no habr\'an lazos ni aristas distintas con los mismos extremos).

\index{gr\'afica!id\'entica}
Decimos que dos gr\'aficas $G$ y $H$ son \textbf{\textit{id\'enticas}} si $V(G) =V(H)$ y $E(G)=E(H)$, y lo denotamos
por $G = H$.
\index{gr\'afica!isomorfas}
Dos gr\'aficas $G$ y $H$ son \textbf{\textit{isomorfas}}, $G\cong H$, si existe una funci\'on biyectiva $\rho : V(G)
\to V(H)$  tal que $xy \in E(G)$ si y s\'olo si $\rho(x)\rho(y) \in E(H)$

\begin{figure}[ht]
    \begin{center}
        \begin{tikzpicture}
        [every circle node/.style ={circle,draw,minimum size= 5pt,inner sep=0pt, outer sep=0pt},
            every rectangle node/.style ={}]
            ;

            \begin{scope}[yshift=-0.5cm, scale=0.5]
                \node [circle] (1) at (90:3)[label=90:$v_1$]{};
                \node[circle] (2) at (210:3)[label=135:$v_2$]{};
                \node[circle] (3) at (330:3)[label=45:$v_3$]{};
                \node [circle] (4) at (4.5,2)[label=270:$v_4$]{};
                \node (g) at (-4.5,4)[]{$G$};


                %\draw [-, shorten <=3pt, shorten >=3pt, >=stealth, line width=.7pt] (2) to node[below] {$e_2$} (3);
                %\draw [-, shorten <=3pt, shorten >=3pt, >=stealth, line width=.7pt] (2) to node[left] {$e_1$} (1);
                %\draw [-, shorten <=3pt, shorten >=3pt, >=stealth, line width=.7pt] (1) to node[right] {$e_3$} (3);
                %\draw [-, shorten <=3pt, shorten >=3pt, >=stealth, line width=.7pt] (1) to node[above] {$e_4$} (4);
            \end{scope}

            \begin{scope}[xshift=7.5cm, scale=0.5]
                \node [circle] (1) at (270:3)[label=270:$v$]{};
                \node[circle] (2) at (30:3)[label=45:$u$]{};
                \node[circle] (3) at (150:3)[label=135:$w$]{};
                \node [circle] (4) at (0,0)[label=90:$x$]{};
                \node (g) at (-4.5,3.5)[]{$G'$};

                %\draw [-, shorten <=3pt, shorten >=3pt, >=stealth, line width=.7pt] (2) to [bend right=30]
                %node[above] {$b$} (3);
                %\draw [-, shorten <=3pt, shorten >=3pt, >=stealth, line width=.7pt] (2) to [bend left=30]
                %node[right] {$a$} (1);
                %\draw [-, shorten <=3pt, shorten >=3pt, >=stealth, line width=.7pt] (1) to [bend left=30] node[left]
                %    {$c$} (3);
                %\draw [-, shorten <=3pt, shorten >=3pt, >=stealth, line width=.7pt] (1) to node[left] {$d$} (4);
            \end{scope}


        \end{tikzpicture}
    \end{center}\caption{ Podemos observar que la gr\'afica $G$ es isomorfa a la gr\'afica $G'$ }
\end{figure}



\index{gr\'afica!completa}
Una \textbf{\textit{gr\'afica completa}} es una gr\'afica simple en la cual cada par de v\'ertices distintos se une
por una arista.
\index{gr\'afica!bipartita}
Una \textbf{\textit{gr\'afica bipartita}} es aquella en la cual sus v\'ertices se pueden dividir en dos conjuntos $X$
y $Y$, ajenos y no vacios, de forma que cada arista tiene un extremo en $X$ y otro en $Y$.
Entonces $(X,Y)$ se llama bipartici\'on de la gr\'afica.
\index{gr\'afica!bipartita completa}
Una \textbf{\textit{gr\'afica bipartita completa}} es una gr\'afica bipartita simple con una bipartici\'on $(X,Y)$ en
la que cada v\'ertice de $X$ se une con cada v\'ertice de $Y$.

Si \big|$X\big|=m$ \  \  y \  \  \big|$Y\big|=n$ entonces esta gr\'afica se denota $ K_{m,n}$.


\begin{figure}[ht]
    \begin{center}
        \begin{tikzpicture}
        [every circle node/.style ={circle,draw,minimum size= 5pt,inner sep=0pt, outer sep=0pt},
            every rectangle node/.style ={}]
            ;

            \begin{scope}[xshift=-5cm, scale=0.5]
                \node [circle] (1) at (0:0)[label=180:$v_1$]{};
                \node (g) at (0,3.5)[]{$K_1$};
            \end{scope}

            \begin{scope}[xshift=-3cm, scale=0.5]
                \node [circle] (1) at (0:1.5)[label=0:$v_1$]{};
                \node [circle] (2) at (180:1.5)[label=180:$v_2$]{};
                %\foreach \from/\to in {1/2}
                %\draw [-, shorten <=3pt, shorten >=3pt, >=stealth, line width=.7pt] (\from) to  (\to);
                \node (g) at (0,3.5)[]{$K_2$};
            \end{scope}

            \begin{scope}[xshift=0cm, scale=0.5]
                \node [circle] (1) at (90:2)[label=180:$v_1$]{};
                \node [circle] (2) at (210:2)[label=210:$v_2$]{};
                \node [circle] (3) at (330:2)[label=330:$v_3$]{};
                %\foreach \from/\to in {1/2,1/3,2/3}
                %\draw [-, shorten <=3pt, shorten >=3pt, >=stealth, line width=.7pt] (\from) to  (\to);
                \node (g) at (0,3.5)[]{$K_3$};
            \end{scope}

            \begin{scope}[xshift=3cm, scale=0.5]
                \node [circle] (1) at (90:2)[label=180:$v_1$]{};
                \node [circle] (2) at (180:2)[label=180:$v_2$]{};
                \node [circle] (3) at (270:2)[label=270:$v_3$]{};
                \node [circle] (4) at (360:2)[label=360:$v_4$]{};
                %\foreach \from/\to in {1/2,1/3,1/4,2/3,2/4,3/4}
                %\draw [-, shorten <=3pt, shorten >=3pt, >=stealth, line width=.7pt] (\from) to  (\to);
                \node (g) at (0,3.5)[]{$K_4$};
            \end{scope}

            \begin{scope}[xshift=6cm, scale=0.5]
                \node [circle] (1) at (72:2)[label=72:$v_1$]{};
                \node [circle] (2) at (144:2)[label=144:$v_2$]{};
                \node [circle] (3) at (216:2)[label=216:$v_3$]{};
                \node [circle] (4) at (288:2)[label=288:$v_4$]{};
                \node [circle] (5) at (360:2)[label=360:$v_5$]{};
                %\foreach \from/\to in {1/2,1/3,1/4,1/5,2/3,2/4,2/5,3/4,3/5,4/5}
                %\draw [-, shorten <=3pt, shorten >=3pt, >=stealth, line width=.7pt] (\from) to  (\to);
                \node (g) at (0,3.5)[]{$K_5$};
            \end{scope}
        \end{tikzpicture}
    \end{center}\caption{Gr\'aficas completas de orden menor o igual a 5}
\end{figure}





\begin{figure}[ht]
    \begin{center}
        \begin{tikzpicture}
        [every circle node/.style ={circle,draw,minimum size= 5pt,inner sep=0pt, outer sep=0pt},
            every rectangle node/.style ={}]
            ;

            \begin{scope}[xshift=0cm, scale=0.7]
                \node [circle] (x_1) at (0,0)[label=180:$x_1$]{};
                \node [circle] (x_2) at (0,2)[label=180:$x_2$]{};
                \node [circle] (x_3) at (0,4)[label=180:$x_3$]{};
                \node [circle] (x_4) at (0,6)[label=180:$x_4$]{};
                \node [circle] (y_1) at (5,1)[label=0:$y_1$]{};
                \node [circle] (y_2) at (5,3)[label=0:$y_2$]{};
                \node [circle] (y_3) at (5,5)[label=0:$y_3$]{};

                %\foreach \from/\to in {x_1/y_1,x_1/y_2,x_1/y_3,x_2/y_1,x_2/y_2,x_2/y_3,x_3/y_1,x_3/y_2,x_3/y_3,
                %x_4/y_1,x_4/y_2,x_4/y_3}
                %\draw [-, shorten <=3pt, shorten >=3pt, >=stealth, line width=.7pt] (\from) to  (\to);

            \end{scope}
        \end{tikzpicture}
    \end{center}\caption{Gr\'afica bipartita completa $K_{4,3}$ }
\end{figure}


\begin{definicion}
    \index{camino}     Un\textbf{\textit{ camino}} en $G$ es una sucesi\'on finita $W= v_0e_1v_1e_2v_2, \dots ,
    e_kv_k$ de v\'ertices y aristas alternadas, donde $1 \le i \le k$. Los extremos de $e_i$ son $v_{i-1}$ y $v_i$ y
    decimos que $W$ es un camino de $v_0$ a $v_k$. Llamamos a $v_0$ y a $v_k$ inicio y t\'ermino de $W$
    respectivamente, a los dem\'as v\'ertices de $W$, $v_1, v_2, \dots,v_{k-1}$, los conocemos como v\'ertices
    internos. La \textbf{\textit{longitud}} \index{longitud} de $W$ es el entero $k$. En una gr\'afica simple un
    camino $W = v_0e_1v_1e_2v_2, \dots ,e_kv_k$ se determina por una sucesi\'on $v_0, v_1, \dots, v_k$ de los
    v\'ertices de $W$, por tanto un camino en una gr\'afica simple se puede expresar simplemente por su sucesi\'on de
    v\'ertices.
    Si las aristas $e_1,e_2,e_3,..., e_k$ de un camino son distintas entonces decimos que es un
    \textbf{\textit{paseo}}\index{camino!paseo}.
    Si los v\'ertices $v_1,v_2,v_3,...,v_k$ de un camino son distintos entonces la llamamos
    \textbf{\textit{trayectoria}}\index{camino!trayectoria},
    donde $v_1$ y $v_k$ son los extremos de la trayectoria.
    Un \textbf{\textit{camino cerrado}}\index{camino!camino cerrado} $\mathscr{C}$, es un camino en el cual los
    extremos son el mismo v\'ertice, $\mathscr{C} = v_0 v_1 \dots v_k=v_0$. Un
    \textbf{\textit{ciclo}}\index{camino!ciclo} es un camino cerrado que no repite v\'ertices salvo el primero y el
    \'ultimo.


    \begin{figure}[ht]
        \begin{center}
            \begin{tikzpicture}
            [every circle node/.style ={circle,draw,minimum size= 5pt,inner sep=0pt, outer sep=0pt},
                every rectangle node/.style ={}]
                ;
                \begin{scope}[xshift=6cm, scale=0.7]

                    \node [circle] (1) at (72:2)[label=72:$v_1$]{};
                    \node [circle] (2) at (144:2)[label=144:$v_2$]{};
                    \node [circle] (3) at (216:2)[label=216:$v_3$]{};
                    \node [circle] (4) at (288:2)[label=288:$v_4$]{};
                    \node [circle] (5) at (360:2)[label=360:$v_5$]{};
                    %\foreach \from/\to in {1/2,1/3,1/4,1/5,2/3,2/4,2/5,3/4}
                    %\draw [-, shorten <=3pt, shorten >=3pt, >=stealth, line width=.7pt] (\from) to  (\to);
                    %\draw [-, shorten <=3pt, shorten >=3pt, >=stealth, line width=.7pt] (5) to [bend left=50] (4);
                    \node (g) at (150:5){$G$};
                \end{scope}
            \end{tikzpicture}
        \end{center}\caption{Una gr\'afica $G$ donde podemos ver el camino $v_3v_1v_4v_2v_5v_1v_4v_2$, el paseo
            $v_1v_3v_2v_1v_4$, la trayectoria $v_1v_4v_5v_2v_3$, el camino cerrado $v_1v_3v_2v_1v_5v_4v_1$ y el ciclo
            $v_4v_5v_1v_2v_4$.}
    \end{figure}


\end{definicion}

Dados $x,y \in V(G)$ una $xy$-\emph{trayectoria}\index{camino!trayectoria!xy-trayectoria} en $G$ es una trayectoria
en $G$ con extremos $x,y$. La \textbf{\textit{distancia}}\index{longitud!distacia} entre $x$ y $y$ es la longitud de
la $xy$-trayectoria mas corta en $G$ y se denota $d_G(x,y)$, cuando no haya lugar a confusiones lo escribiremos $d(x,
y)$. El \textbf{\textit{di\'ametro}}\index{longitud!di\'ametro} de $G$ es la distancia m\'axima entre dos v\'ertices
de \emph{G}.

\begin{definicion}
    Una gr\'afica $G$ es \textbf{\textit{conexa}}\index{gr\'afica!conexa} si para todo par de v\'ertices $x,y \in V
    (G)$ existe un $xy$-\emph{camino} en $G$. Una componente conexa es una subgr\'afica conexa m\'axima por
    contenci\'on.
\end{definicion}

El grado de un v\'ertice $v$, $d_G(v)$, en una gr\'afica $G$ es el n\'umero de aristas de $G$ que inciden en $v$.
Denotamos al grado m\'inimo de $G$ como $\delta(G)$ y al grado m\'aximo de $G$, $\Delta(G)$.

\begin{proposicion}
    Sea $G$ una gr\'afica, todo $xy$-camino en $G$ contiene una $xy$-trayectoria en $G$.
\end{proposicion}


\begin{proof}
    Por inducci\'on sobre la longitud del camino. Cuando la longitud del camino $W$, es uno tenemos una arista en $W$
    . Cuando la longitud del camino $W$ es $n$ suponemos que hay una $xy$-trayectoria en W. Sea $\mathscr{C} =
    (x=v_0, v_1, \dots, v_n,  v_{n+1}=y)$ un camino de longitd $n+1$; Consideremos $\mathscr{C'} = (x, v_1, \dots,
    v_n)$, camino de longitud $n$ que por hip\'otesis inductiva ya contiene una $xv_n$-trayectoria, $P$. Ahora veamos
    que si $P$ contiene a $y$ quiere decir que existe una $xy$-trayectoria en $P \subseteq \mathscr{C'} \subseteq
    \mathscr{C}$. Si $P$ no contiene a $y$ entonces $P \cup v_ny \subseteq \mathscr{C}$ es una $xy$-trayectoria.
\end{proof}

\begin{proposicion}
    Si  $G$ es una gr\'afica en la que existen dos $uv$-trayectorias distintas entonces la uni\'on de estas
    trayectorias contiene un ciclo.
\end{proposicion}


\begin{proof}
    Sean $P_1=(u=x_1,\dots,x_m=v)$, y $P_2=(u=y_1,\dots,y_n=v)$ dos $uv$-trayectorias distintas, consideremos a
    $r=\min\{i\in\mathbb{N}\ |\ x_i\neq y_i\}$, dicho $r$ existe pues $P_1\neq P_2$, y tomemos a
    $s=\min\{j\in\mathbb{N}\ |\ j>k, x_j\in P_2\}$, es decir $x_s=y_t$ para alg\'un $r<t\leq n$, observemos que $s$
    existe puesto que $x_m=y_n=v$.

    Por lo tanto $x_{r-1}P_1x_s$ y $y_{r-1}P_2y_t$  son dos trayectorias que se intersectan solamente en sus
    extremos, entonces $x_sP_1x_{r-1}\cup y_{r-1}P_2y_t=(x_s,\dots,x_{r-1}=y_{r-1},\dots,y_t=x_s)$ es un ciclo.
\end{proof}

Para demostrar la pr\'oxima proposi\'on utilizaremos el siguiente lema.

\begin{lema}
    \label{lem1}
    Todo camino cerrado de longitud impar contiene un ciclo de longitud impar.
\end{lema}

\begin{proof}
    Demostraci\'on por inducci\'on sobre el n\'umero de v\'ertices que se repiten en el camino.
    Sea $W = (v_1,v_2, \dots, v_k)$, un camino cerrado de longitud impar el cual no repite v\'ertices, entonces $W$
    es un ciclo de longitud impar.
    Supongamos que para todo camino cerrado de longitud impar con $k$ v\'ertices repetidos, donde $0\le k \le n$,
    contiene un ciclo de longitud impar.
    Sea $\mathscr{C}$ un camino cerrado de longitud impar con n v\'ertices repetidos,$\mathscr{C}= (x_1,x_2, ... ,
    x_r)$.
    Sin perdida de generalidad existe $j$ tal que $1<j<r$ con $x_1 = x_j$, lo que nos induce a dos caminos cerrados
    $\mathscr{C}_1$ y $\mathscr{C}_2$,
    $\mathscr{C}_1= (x_1, ..., x_j)$ y
    $\mathscr{C}_2= (x_j, ..., x_r)$,
    de estos dos caminos uno tiene longitud par y otro longitud impar, adem\'as de que en cada uno se repiten menos
    de $n$ v\'ertices.
    Entonces, sin perdida de generalidad supongamos que la longitud de $\mathscr{C}_1$ es impar, por hipotesis
    inductiva sabemos que contiene un ciclo de longitud impar.
\end{proof}


\begin{proposicion}
    Si $G$ es una gr\'afica con al menos dos v\'ertices, entonces es bipartita si y s\'olo si no contiene ciclos
    impares.
\end{proposicion}


\begin{proof}
    Sea $G$ una gr\'afica bipartita y $\mathscr{C} = (x_1,x_2, ..., x_r)$ un ciclo en $G$.
    Sea $V_1$ y $V_2$ la bipartici\'on de $V(G)$, v\'ertices de $G$, vamos a suponer sin perdida de generalidad que
    $x_1 \in V_1$ como $x_1x_2 \in E(G)$ por definici\'on de gr\'afica bipartita $x_2 \in V_2$, como $x_2x_3 \in E(G)
    $ por definici\'on de gr\'afica bipartita $x_3 \in V_1$, en general $x_{2i-1} \in V_1$ y $x_{2i} \in V_2$. Veamos
    que $x_rx_1 \in E(G)$ entonces $x_r \in V_2$ entonces $r$ es par y por tanto $\mathscr{C}$ es par, por tanto $G$
    no contiene ciclos de longitud impar.

    Sea $x \in V(G)$. Definamos a los conjuntos

    $$N_1=\left\{ y \in V(G)\ \ \textnormal{tal que} \ \ d_G(x,y) \textnormal{ es par}\right\}$$
    $$N_2=\left\{ y \in V(G)\ \ \textnormal{tal que} \ \ d_G(x,y) \textnormal{ es impar}\right\}$$


    Observemos que $ N_1 \cup N_2  =  V(G)$,
    $ N_1 \cap N_2  =  \varnothing$,
    $N_1 \ne \varnothing \ne N_2$.
    Supongamos que existe $uw \in E(G)$ con extremos en $N_1$, es decir, $u \in N_1$ y $w \in N_1$, entonces la
    distancia de $x$ a $u$ y de $x$ a $w$ es par. Sea $\mathscr{C}_1$ una $xu$-trayectoria de longitud par y
    $\mathscr{C}_2$ una $xw$-trayectoria de longitud par veamos entonces que $\mathscr{C}_1 \cup uw \cup
    \mathscr{C}_2^{-1}$ es un camino cerrado de longitud impar, entonces, por el lema \ref{lem1} hay un ciclo de
    longitud impar, por lo tanto hay una contradicci\'on, ya que $G$ no contiene ciclos impares. An\'alogo para el
    caso en el que tomamos a $uw \in E(G)$ con extremos en $N_2$, ya que vamos a obtener dos caminos de longitud
    impar que al sumarle la arista $uw$ nos da un camino cerrado de longitud impar por lo que entonces hay un ciclo
    de longitud impar, implicando una contradicci\'on.
\end{proof}

\index{gr\'afica!ac\'iclica}
\index{\'arbol}
Una gr\'afica \textbf{ac\'iclica} es aquella que no contiene ciclos. Un \textbf{\'arbol} es una gr\'afica conexa
ac\'iclica.


\begin{figure}[ht]
    \begin{center}
        \begin{tikzpicture}
        [every circle node/.style ={circle,draw,minimum size= 5pt,inner sep=0pt, outer sep=0pt},
            every rectangle node/.style ={}]
            ;
            \begin{scope}[xshift=6cm, scale=0.7]

                \node [circle] (1) at (72:2)[label=72:$v_1$]{};
                \node [circle] (2) at (144:2)[label=144:$v_2$]{};
                \node [circle] (3) at (216:2)[label=216:$v_3$]{};
                \node [circle] (4) at (288:2)[label=288:$v_4$]{};
                \node [circle] (5) at (360:2)[label=360:$v_5$]{};
                %\foreach \from/\to in {2/1,2/3,2/4,2/5}
                %\draw [-, shorten <=3pt, shorten >=3pt, >=stealth, line width=.7pt] (\from) to  (\to);
                \node (g) at (150:5){$T$};
            \end{scope}
        \end{tikzpicture}
    \end{center}\caption{\'Arbol $T$, gr\'afica conexa ac\'iclica.}
\end{figure}

\begin{teorema}
    \label{ATU}
    En un \'arbol $T$, cualesquiera dos v\'ertices estan conectados por una $xy$-trayectoria \'unica en $T$.
\end{teorema}

\begin{proof}
    Por contradicci\'on.
    Supongamos que existen $x,y \in V(T)$ y que existen $P_1$ y $P_2$ $xy$-trayectorias (distintas entre si) en $T$.
    Sabemos que existe $e= v_1v_2$,  $e \in E(P_1)$ tal que $e \notin E(P_2)$, Podemos ver que $P_1 \cup P_2-e$ es
    conexa, entonces existe una $v_1v_2-trayectoria$ $P_3$ y entonces $P_1 \cup P_2$ contiene un ciclo, $P_3 \cup
    \left\{e\right\}$, pero $P_1 \cup P_2$ es subgr\'afica de $T$, y $T$ es ac\'iclica, por tanto es una
    contradicci\'on.
\end{proof}

Si $T$ es un \'arbol una hoja de $T$ es un v\'ertice $v \in V(T)$ tal que $d(v) = 1$.

\begin{teorema}
    Todo \'arbol tiene al menos dos hojas.
\end{teorema}

\begin{proof}
    Sea $T$ un \'arbol no trivial, entonces $d_T(v) \ge 1$ para todo $v \in V(T)$. Sea $P$ una $xy$-trayectoria de
    longitud m\'axima en $T$, con extremos $x$ y $y$, $T$ no trivial afirmamos que $x$ y $y$ son hojas, pues de no
    ser as\'i suponemos sin perdida de generalidad que $d_T(v) \ge 2$ entonces existe $w \in N_G(x)$ tal que no esta
    en $P$, entonces $\left\{wx\right\} \cup P$ es una trayectoria de mayor longitud que $P$, o, existe $w \in N_T(x)
    $ que est\'a en $P$ y entonces existe un ciclo, contradiciendo que $T$ es ac\'iclica.
\end{proof}

\begin{teorema}
    Si $T$ es un \'arbol, entonces $|E(T)|=|V(T)|-1$.
\end{teorema}

\begin{proof}
    Por inducci\'on sobre $|V(T)|$, el orden de $T$. Si $T$ tiene un s\' olo v\'ertice, el resultado se cumple
    trivialmente.

    Supongamos el resultado v\' alido para todo \'arbol con a lo m\' as $n$ v\'ertices. Sean $T$ un \'arbol con $n+1$
    v\'ertices y $uv $ una arista de $T$. Por el Teorema \ref{ATU}, no existe una $uv$-trayectoria en $T$ y por lo
    tanto $G - uv$ tiene dos componentes conexas $T_1$ y $T_2$. Como $T_1$ y $T_2$ son ac\' iclicas, resultan ser
    \'arboles, y por hip\'otesis inductiva $$|E(T_i)| = |V(T_i)| -1, \ \ i \in \{ 1, 2 \}$$ pero claramente $$|E(T)|
    = |E(T_1)| + |E(T_2)| + 1 = |V(T_1)|  + |V(T_2)| - 1 = |V(T)| - 1.$$
\end{proof}


\section{Hipercubos}

Si $G$ y $H$ son gr\' aficas, el \textbf{producto cuadro}, $G \Box H$, se define como la gr\' afica que tiene por
conjunto de v\' ertices al producto cartesiano $G \times H$ y donde $$(g,h)(g', h') \in A(G \Box H) \textnormal{ si y
s\'olo si }\left\{ \begin{array}{c}
                       g = g' \textnormal{ y } hh' \in E(H) \\ \textnormal{\' o } \\ gg' \in E(G) \textnormal{ y } h
                       = h'
\end{array} \right..$$

A partir de la definici\' on de producto cuadro podemos hacer una observaci\'on muy \'util. Fijemos $g \in V(G)$ y
llamemos $H_g$ a la subgr\' afica de $G \Box H$ inducida por el conjunto $\{ (g, h) \colon\ h \in V(H) \}$. Como la
primera coordenada de todos los v\'ertices en $H_g$ es $g$, resulta claro que $H_g \cong H$; el isomorfismo est\'a
dado por la funci\'on $\varphi (g,h) = h$. An\'alogamente, si fijamos $h \in V(H)$ y llamamos $G_h$ a la subgr\'
afica de $G \Box H$ inducida por el conjunto $\{ (g, h) \colon\ g \in V(G) \}$, entonces $G_h \cong G$. As\' i, por
cada v\' ertice de $H$, tenemos una copia isomorfa de $G$ y por cada v\' ertice de $G$ tenemos una copia isomorfa de
$H$ en $G \Box H$.

Esta observaci\' on puede generalizarse para el caso en el que tengamos el producto cuadro de un n\' umero finito de
gr\' aficas. Denotaremos por $\prod_{i = 1}^n G_i$ al producto cuadro de una familia de gr\' aficas $\{ G_1, \dots,
G_n \}$. Se sigue de la definici\'on de producto cuadro que $(u_1, \dots, u_n)(v_1, \dots, v_n) \in E \left( \prod_{i
= 1}^n G_i \right)$ si y s\' olo si existe  $1 \le j \le n$ tal que $u_j v_j \in E(G_j)$  y  $u_i = v_i$ para cada $i
\ne j$. A partir de esta observaci\'on, puede inferirse la siguiente proposici\'on.

\begin{proposicion}
    \label{prodcart}
    Sean $\{ G_1, \dots, G_n \}$ una familia de gr\' aficas, $I \subseteq \{ 1, \dots, n \}$, $J = \{ 1, \dots, n \}
    \setminus I$, y $\{ g_i \}_{i \in I}$ un conjunto de v\' ertices tales que $g_i \in V(G_i)$ para cada $i \in I$.
    Si $X = \left\{ (v_1, \dots, v_n) \in V(\prod_{i=1}^n G_i) \colon\ v_i = g_i \textnormal{ si } i \in I, v_j \in
    G_j \textnormal{ si } j \notin I \right\}$, entonces $$\left( \prod_{i=1}^n G_i \right) [ X ] \cong \prod_{j \in
    J} G_j.$$   Como caso particular, si $I =  \{ i \}$, $$\left( \prod_{i=1}^n G_i \right) [ X ] \cong \prod_{j \ne
    i} G_j.$$
\end{proposicion}

La demostraci\' on se omite por ser sencilla pero bastante t\' ecnica, basta observar que el isomorfismo est\' a dado
por la proyecci\' on sobre el conjunto de coordenadas $J$.

Consideremos las siguientes tres familias de gr\'aficas.

\begin{definicion}
    El \emph{hipercubo}\index{hipercubo} $n$-\emph{dimensional}, $H_n$ ($n \ge 1$), es la gr\'afica que tiene como
    conjunto de v\'ertices al conjunto de todas las $n$-adas ordenadas de 0's y 1's, donde dos $n$-adas ordenadas son
    adyacentes si y s\'olo si difieren en exactamente una coordenada.
\end{definicion}

\begin{definicion}
    Definimos a la familia $R_n$ recursivamente:
    \begin{itemize}
        \item $R_1  =  K_2$,
        \item $R_{n+1}  =  R_n \Box K_2$.
    \end{itemize}
\end{definicion}

\begin{definicion}
    Los v\'ertices de $S_n$ son los elementos de la potencia de $A$, donde $A=\left\{1,2,3, \dots, n\right\}$; $XY$
    est\'a en las aristas de $S_n$ si y s\'olo si $|X \Delta Y|=1$.
\end{definicion}

En esta secci\'on demostraremos que estas tres familias de gr\'aficas son isomorfas, por lo tanto, tendremos tres
definiciones distintas para trabajar con los hipercubos $d$-dimensionales.


\begin{lema}
    \label{lemcub}
    En $S_n$ $|X \Delta Y|=1$ si y s\'olo si existe un \'unico $j \in \left\{1,2,3, \dots, n\right\}$ tal que $\chi_j
    (X) \ne \chi_j(Y)$, donde $\chi_i: \mathcal{P}(A_n) \to \left\{0,1\right\}$ tal que  $\chi_i(Y) = \left\{
    \begin{array}{cl}
        1 & \textnormal{ si } i \in Y \\ 0 & \textnormal{ si } i \notin Y.
    \end{array} \right.$
\end{lema}

\begin{proof}
    Recordemos que $X \Delta Y=(X \setminus Y) \cup (Y \setminus X)$, entonces $|X \Delta Y|=1$ si y s\'olo si $|(X
    \setminus Y) \cup (Y \setminus X)|=1$. Sabemos que $(X \setminus Y) \cap (Y \setminus X)= \varnothing$, por lo
    que $|(X \setminus Y) \cup (Y \setminus X)|=|(X \setminus Y)| + |(Y \setminus X)|=1$.
    Sin perdida de generalidad $|X \setminus Y|=1$ y $|Y \setminus X|=0$, entonces existe un \'unico $z \in X
    \setminus Y$ y $Y\subseteq X$ por lo tanto $\chi_z(X)=1$, $\chi_z(Y)=0$ y $Y \subseteq X$, entonces $\chi_i(X)
    =\chi_i(Y)$ para todo $i \ne z$.

    Si existe un \'unico $1 \le k \le n$ tal que $\chi_k(X) \ne \chi_k(Y)$, supongamos sin perdida de generalidad que
    $\chi_k(X)=1$ y que $\chi_k(Y)=0$ entonces $k \in X$ y $k \notin Y$ infiriendo que $k \in X \setminus Y$. Ahora
    vemos que $Y \setminus X= \varnothing$, de lo contrario vemos que existe $k'$ tal que $k' \in Y$ y $k' \notin X$,
    claramente $k' \ne k$ y $\chi_{k'}(X) \ne \chi_{k'}(Y)$ llegando a una contradicci\'on. Por tanto $Y \setminus
    X=\varnothing$ y por la unicidad de $k$ se cumple que $|X \setminus Y|=1$. Entonces $|(X \setminus Y) \cup (Y
    \setminus X)|=1$ y por lo tanto $|X \Delta Y|=1$.
\end{proof}

\begin{teorema}
    Para cada entero $n \ge 1$ se cumple \[H_n \cong R_n \cong S_n.\]
\end{teorema}


\begin{proof}
    Veamos primero que $S_n \cong H_n$.

    Como $A=\left\{1,2,3, \dots, n\right\}$, para cada $i \in A$ definimos  $\chi_i: \mathcal{P}(A_n) \to \left\{0,
    1\right\}$ tal que  $$\chi_i(Y) = \left\{ \begin{array}{cl}
                                                  1 & \textnormal{ si } i \in Y \\ 0 & \textnormal{ si } i \notin Y.
    \end{array} \right.$$

    Sea $\varphi: \mathcal{P}(A_n) \to \left\{0,1\right\}^n$ tal que $$\varphi(Y)=(\chi_n(Y), \chi_{n-1}(Y), \dots,
    \chi_1(Y)),$$ entonces $\varphi(Y) \in V(H_n)$ y resulta claro que $\varphi$ est\'a bien definida.

    Para cada $i \in A$, definimos $\zeta_i: \left\{0,1\right\} \to \mathcal{P}(A_n) $ por  $$\zeta_i(x) = \left\{
    \begin{array}{cl}
        \left\{i\right\} & \textnormal{ si } x=1 \\ \varnothing & \textnormal{ si } x=0
    \end{array} \right.;$$  si $y=(k_n, k_{n-1}, \dots, k_1)$ entonces podemos definir a  $\psi:\left\{0,
    1\right\}^{n} \to \mathcal{P}(A_n)$ de la siguiente forma: $$\psi(y)=\zeta_n(k_n) \cup\zeta_{n-1}(k_{n-1}) \cup
    \dots \cup \zeta_1(k_1) = \bigcup_{i=1}^n \zeta_i(k_i).$$

    Es f\'acil observar que: $$\zeta_i (\chi_i(Y)) =  \left\{ \begin{array}{cl}
                                                                  \left\{ i \right\} & \textnormal{ si } i \in Y    \\
                                                                  \varnothing        & \textnormal{ si } i \notin Y
    \end{array} \right.;$$  de esta manera,
    \begin{align*}{rCl}
        \psi \circ \varphi(Y) &= \psi(\chi_n(Y), \chi_{n-1}(Y), \dots, \chi_1(Y)) \\
        &= \zeta_n (\chi_n(Y)) \cup \zeta_{n-1}(\chi_{n-1}(Y)) \cup \dots \cup \zeta_1(\chi_1(Y)) \\
        &= Y,
    \end{align*}
    por lo tanto, $\psi \circ \varphi = 1_{\mathcal{P}(A_n)}$.

    Por otro lado, si $y \in \{ 0, 1 \}^n$,  entonces $k_j=1$ si y s\'olo si $j \in \bigcup_{i=1}^n \zeta_i(k_i)$ si
    y s\'olo si          $\chi_j(\bigcup_{i=1}^n \zeta_i(k_i))=1$. Adem\'as, $k_j=0$ si y s\'olo si $j \notin
    \bigcup_{i=1}^n \zeta_i (k_i)$ si y s\'olo si $\chi_j(\bigcup_{i=1}^n \zeta_i(k_i))=0$. Por lo tanto,
    \begin{align*}{rCl}
        \varphi \circ \psi(y) &= \varphi(\zeta_n (k_n) \cup \zeta_{n-1}(k_{n-1}) \cup \dots \cup \zeta_1(k_1)) \\
        &= (\chi_n (\bigcup_{i=1}^n \zeta_i(k_n)), \chi_{n-1}(\bigcup_{i=1}^n \zeta_i(k_{n-1})), \dots,
        \chi_1(\bigcup_{i=1}^n   \zeta_i(k_1))) \\
        &= (k_n, k_{n-1}, \dots, k_1).
    \end{align*}
    De esta manera podemos concluir que $\varphi \circ \psi = 1_{\{ 0, 1 \}^n}$. As\'i, obtenemos que $\varphi$ es
    una funci\'on biyectiva. Para demostrar que $\varphi$ es un isomorfismo, observemos que $XY \in E(S_n)$ si y
    s\'olo si $|X \Delta Y|=1$. Por el Lema \ref{lemcub}, \'esto sucede si y s\'olo si existe un \'unico $j \in
    \left\{1,2,3, \dots, n\right\}$ tal que $\chi_j(X) \ne \chi_j(Y)$ si y s\'olo si $\varphi(X) \varphi(Y) \in E
    (H_n)$; por lo tanto $\varphi$ es un isomorfismo entre $H_n$ y $S_n$.

    Veamos ahora que $R_n \cong H_n$. Sabemos que $R_1=K_2$, donde $V(R_1)=\left\{u,v\right\}$ y $E(R_1)
    =\left\{uv\right\}$. Sabemos tambi\' en que $V(H_1)=\left\{(0),(1)\right\}$ y $E(H_1)=\left\{(0)(1)\right\}$.
    Podemos definir a $\varphi_1:V(R_1) \to V(H_1)$ por  $$\varphi_1(u)=0, \varphi_1(v)=1.$$   Resulta claro que
    $\varphi_1$ es una funci\' on biyectiva con inversa $\varphi_1^{-1}:V(H_1) \to V(R_1)$ dada por $\varphi_1^{-1}
    (0)=u, \varphi_1^{-1}(1)=v$. M\' as a\' un, como $E(R_1) = \{ uv \}$ y $E(H_1) = \{ (0)(1) \}$, la biyecci\'on
    $\varphi$ es un isomorfismo.

    Entonces, para cada entero $n \ge 2$ podemos definir a $\varphi_n:V(R_n) \to V(H_n)$ como:   $$\varphi_n(x_0,x_1,
    ..., x_n)=(\varphi_1(x_0),\varphi_1(x_1), ..., \varphi_1(x_n)).$$ Resulta claro que $\varphi_n$ est\'a bien
    definida y que tiene por inversa a la funci\'on $\varphi_n^{-1}:V(H_n) \to V(R_n)$ dada por:
    $$\varphi_n^{-1}(y_0,y_1, ..., y_n)=(\varphi_1^{-1}(y_0),\varphi_1^{-1}(y_1), \dots , \varphi_1^{-1}(y_n)).$$
    Por lo tanto $\varphi_n$ es una funci\'on biyectiva, y basta entonces verificar que preserva las aristas.
    Demostraremos por inducci\'on sobre $n$ que $xy \in E(R_n)$ si y s\'olo si $\varphi_1(x)\varphi_1(y) \in E(H_n)$.

    Para $n=1$ ya lo hemos verificado. Ahora supongamos que $\varphi_{n-1}$ es un isomorfismo entre $R_{n-1}$ y $
    H_{n-1}$ y sean $x = (x_0,x_1, \dots, x_n)$ y $z = (z_0, z_1, \dots, z_n)$. Sabemos que $R_n=R_{n-1} \Box K_2$
    por lo tanto $V(R_n)=X \sqcup Y$, donde $$X_1=\left\{(x_0,x_1, ..., x_{n-1}, u)\big| x_i \in \left\{u,v\right\},
    1 \le i \le n-1 \right\}$$ $$Y_1=\left\{(x_0,x_1, ..., x_{n-1}, v)\big| x_i \in \left\{u,v\right\}, 1 \le i \le
    n-1 \right\}.$$   Claramente $X_1 \cap Y_1 = \varnothing$; adem\'as la Proposici\'on \ref{prodcart} garantiza que
    $R_n[X_1] \cong R_{n-1}$ y $R_n[Y_1] \cong R_{n-1}$.

    An\'alogamente puede observarse que si:
    $$X_2=\left\{(y_0, y_1, \dots, y_{n-1},0) \big| y_i \in \left\{0,1\right\}, 1 \le i \le n-1\right\}$$
    $$Y_2=\left\{(y_0, y_1, \dots, y_{n-1},1) \big| y_i \in \left\{0,1\right\}, 1 \le i \le n-1\right\},$$ entonces
    $X_2 \cap Y_2 = \varnothing$ y $X_2 \cup Y_2= V(H_n)$. Adem\'as, como todos los v\' ertices en $X_2$ tienen la
    misma coordenada final, no es dif\'icil obtener que $H_n [X_2] \cong H_{n-1}$ y, an\' alogamente, $H_n [Y_2]
    \cong H_{n-1}$.

    Por las definiciones de $\varphi_{n-1}$ y $\varphi_n$, resulta claro que $$\varphi_n(x_0,x_1, \dots, x_n) =
    (\varphi_{n-1}(x_0,x_1, \dots, x_{n-1}), \varphi_1(x_n)).$$   Podemos considerar dos casos:

    Si $x_n = z_n$, podemos suponer sin p\'erdida de generalidad que $x_n = z_n = u$. Entonces,  $z,x \in X_1$ y por
    lo tanto $xz \in E(R_n [X_1] )$. Si definimos a $\pi: R_n [X_1] \to R_{n-1}$ por $\pi ( (x_0, \dots, x_{n-1}, u)
    ) = (x_0, \dots, x_{n-1})$ y a $\eta: H_{n-1} \to H [X_2]$ por $\eta ( (w_0, \dots, w_{n-1}) )= (w_0, \dots,
    w_{n-1}, 0)$, resulta claro que \'estos son los isomorfismos antes mencionados, por lo tanto, al definir $\psi =
    \eta \circ \varphi_{n-1} \circ \pi$, y recordando que la composici\'on de isomorfismos es un isomorfismo, tenemos
    que $\psi$ es un isomorfismo. Pero $\psi (x_0, \dots, x_{n-1}, u) = (\varphi_1 (x_0), \dots, \varphi_1 (x_{n-1}),
    0) = \varphi_n (x_0, \dots, x_n)$. As\'i, $\psi = \varphi_n \big|_{R_n [X_1]}$, por lo que el diagrama de la
    Figura \ref{diag} es conmutativo y $\varphi_n \big|_{R_n [X_1]} : R_n [X_1] \to H[X_2]$ es un isormorfismo. En
    particular, $xz \in E(R_n [X_1])$ si y s\' olo si $\varphi_n (x) \varphi_n (z) \in E(H_n [X_2])$ y por lo tanto
    $xz \in E(R_n)$ si y s\'olo si $\varphi(x) \varphi (z) \in E(H_n)$.


    \begin{figure}[H]
        \begin{center}
            \begin{tikzpicture}
%[every circle node/.style ={circle,draw,minimum size= 5pt,inner sep=0pt, outer sep=0pt},
%every rectangle node/.style ={}];
%\begin{scope}[xshift=6cm, scale=0.7]

                \node (1) at (0,0){$R[X_1]$};
                \node (2) at (0,2){$R_{n-1}$};
                \node (3) at (4,2){$H_{n-1}$};
                \node (4) at (4,0){$H[X_2]$};

                %\draw [->, shorten <=1pt, shorten >=1pt, >=stealth, line width=.7pt] (1) to node[left]{$\pi$} (2);
                %\draw [->, shorten <=1pt, shorten >=1pt, >=stealth, line width=.7pt] (2) to
                %node[above]{$\varphi_{n-1}$} (3);
                %\draw [->, shorten <=1pt, shorten >=1pt, >=stealth, line width=.7pt] (3) to node[right]{$\eta$} (4);
                %\draw [->, shorten <=1pt, shorten >=1pt, >=stealth, line width=.7pt] (1) to node[below]{$\varphi_n
                %\big|_{R_n [X_1]}$} (4);
%\end{scope}
            \end{tikzpicture}
        \end{center} \caption{$\varphi_n \big|_{R_n [X_1]} = \psi = \eta \circ \varphi_{n-1} \circ \pi$} \label{diag}
    \end{figure}

    Ahora supongamos $x_n \ne z_n$. Inferimos que $xz \in E(R_n)$ si y s\'olo si $x_i = z_i$, $1 \le i \le {n-1}$, si
    y s\'olo si, sin perdida de generalidad $x=(x_0, x_1,\dots, x_{n-1}, u)$ y $z=(x_0, x_1,\dots, x_{n-1}, v)$ si y
    s\'olo si $\varphi(x)=(\varphi_1(x_0), \varphi_1(x_1),\dots, \varphi_1(x_{n-1}),0)$ y $\varphi(z)=(\varphi_1(x_0)
    , \varphi_1(x_1),\dots, \varphi_1(x_{n-1}),1)$ si y s\'olo si $\varphi(x)\varphi(z) \in E(H_n)$ y $\varphi_1(x_n)
    \ne \varphi_1(z_n)$. Por lo tanto $R_n \cong H_n$ y por lo tanto $S_n \cong H_n \cong R_n$.
\end{proof}

En el hipercubo $d$-dimensional $H_d$ con $n=2^d$ v\'ertices, para cada v\'ertice $x$, hay una etiqueta distinta
asociada con cada una de las aristas incidentes; la etiqueta entre el v\'ertice $x$ y el v\'ertice $z$, es la
posici\'on de la entrada en la que los v\'ertices $x$ y $z$ difieren, llamada dimensi\'on.
Sea $G=H_d=(V,E)$ un hipercubo con $n=|V|$, orden de $G$. Sean $E(x)$ las aristas incidentes a $x \in V$. Sea
$\lambda_x: E(x) \to \left\{1, 2, ..., d\right\}$ una funci\'on inyectiva que define la etiqueta de la arista para el
v\'ertice $x$; es decir, $\lambda_x(x,y)$ indica la entrada en la que $x$ difiere de $y$.


\begin{figure}[ht]
    \begin{center}
        \begin{tikzpicture}
        [every circle node/.style ={circle,draw,minimum size= 5pt,inner sep=0pt, outer sep=0pt},
            every rectangle node/.style ={}]
            ;

            \begin{scope}[xshift=6cm, scale=0.5]
                \node [circle] (1) at (45:2)[label=45:$\text{(1,1)}$]{};
                \node [circle] (2) at (135:2)[label=135:$\text{(0,1)}$]{};
                \node [circle] (3) at (225:2)[label=225:$\text{(0,0)}$]{};
                \node [circle] (4) at (315:2)[label=315:$\text{(1,1)}$]{};
                %\foreach \from/\to in {1/2,2/3,3/4,4/1}
                %\draw [-, shorten <=3pt, shorten >=3pt, >=stealth, line width=.7pt] (\from) to  (\to);
                \node (g) at (150:5){$H_2$};
            \end{scope}

        \end{tikzpicture}
    \end{center}\caption{ Hipercubo 2-dimensional } \label{etiquetas}
\end{figure}






\backmatter

\printindex


\begin{thebibliography}{99}\addcontentsline{toc}{chapter}{Bibliograf\'ia}

\bibitem{Articulo} \label{Articulo}
Paola Flocchini., Miao Jun Huang., Flamina,L.Luccio. ``Capturing an Intruder in the Hypercube by Mobile Agents''. \emph{Discrete Mathematics}, 309(8) (2009) 1971-1985.
 
\bibitem{Bang}  \label{Bang}
Bang-Jensen, J. y Gutin, G. \emph{Graph Theory}. Springer, 2009.

\bibitem{Berlakamp} \label{Berlekamp}
L.Barriere, P. Flocchini, P. Fraigniaud, and N. Santoro. \emph{Capture of an intruder by mobile agents, Proc 14th ACM Symp Parallel Algorithms Architect (SPAA)}.Winnipeg, Manitoba, Canada, 2002, pp. 200-209.

\bibitem{CK}
L.S. Chandran y T. Kavitha, \emph{The treewidth and pathwidth of hypercubes}, Discrete Mathematics 306 (2006), 359--365.

\bibitem{Bondy} \label{Bondy}
Bondy, J.A.  y  Murty, U.S.R. \emph{Graph Theory}. Springer, 2008.

\bibitem{Diestel} \label{Diestel}
Diestel, R. \emph{Graph Theory}. Springer, 2005.

\bibitem{Hijab}  \label{Hilab}
Hijab, O. \emph{Introduction to Calculus and Classical Analysis}. Springer, 2007.

\bibitem{KP}
L.M. Kirousis y C.H. Papadimitriou \emph{Searching and Pebbling}, Theor Comut Sci 47 (1986), 205--218.

\bibitem{Spivak}  \label{Spivak}
Spivak, M. \emph{Calculus}. 3ra ed. Publish or Perish, Inc. 2006.

\end{thebibliography}

\newpage
\mbox{}


\end{document}
